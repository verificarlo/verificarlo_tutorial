\documentclass{TP}
\usepackage[american]{babel}
\usepackage[utf8]{inputenc}
\usepackage[T1]{fontenc}
\usepackage{graphicx}
\usepackage{amsmath}
\usepackage[all]{xy}
\usepackage{hyperref}
\usepackage{todonotes}
\usepackage{placeins}
\usepackage{algorithm}
\usepackage{algpseudocode}

\graphicspath{{images/} {figs/}}

\newcommand*{\matminus}{%
  \leavevmode
  \hphantom{0}%
  \llap{%
    \settowidth{\dimen0 }{$0$}%
    \resizebox{1.1\dimen0 }{\height}{$-$}%
  }%
}

\usepackage{listings}

\lstdefinestyle{customC}{
  belowcaptionskip=1\baselineskip,
  breaklines=true,
  xleftmargin=\parindent,
  language=C,
  showstringspaces=false,
  basicstyle=\tiny\ttfamily,
  keywordstyle=\bfseries\color[rgb]{0.580, 0.000, 0.827},
  %{purple!40!lightgray},
  commentstyle=\itshape\color{green!40!black},
  identifierstyle=\bfseries\color{cyan!75!black},
  stringstyle=\color{orange},
  deletekeywords={double,float},
  classoffset=1, % starting new class
  otherkeywords={double,float},
  morekeywords={double,float},
  keywordstyle=\bfseries\color{green!55!black},
  classoffset=0
}


\usepackage[parfill]{parskip}

\title{\includegraphics[keepaspectratio, scale=0.4]{verificarlo-logo.pdf}\\[4mm]
  Tutorial}

\hypersetup{pdftitle={Tutoriel Verificarlo}}

\author{eric.petit@intel.com; yohan.chatelain@uvsq.fr; pablo.oliveira@uvsq.fr;
  francois.fevotte@edf.fr; bruno.lathuiliere@edf.fr}

\date{Université de Versailles Saint-Quentin en Yvelines - Intel - EDF}

\begin{document}
\maketitle
\tableofcontents
\vspace{2cm}
Verificarlo is an open-source project under GPL v3. It is freely available at the following address : \\
\centerline{\url{https://github.com/verificarlo/verificarlo}}

~\\A Debian package and a Docker container are also available: \\
\begin{itemize}
\item \url{https://github.com/verificarlo/verificarlo/releases/tag/v0.2.0-debian}
\item \url{https://hub.docker.com/r/verificarlo/verificarlo/}
\end{itemize}

\section{Verificarlo Basics}
\subsection{Start a verificarlo project}

{\sc{Verificarlo}} is overloading the LLVM compiler. It will be used instead of your usual compiler using the command {\tt verificarlo}.

To start a project, the first thing is to modify your build system to use the {\tt{verificarlo}} compiler.

Since the fundamental usage of Verificarlo is based on multiple execution sample study, it is necessary to provide easy execution and data of interest collection script for your application. Since all applications have different need, you are responsible to provide {\it ad hoc} implementation. However, you will find some example in the tutorial code and the Verificarlo {\tt test} directory.

\subsection{Dealing with Verificarlo configurations}

\subsubsection{From a terminal}

Verificarlo is controlled through three environment variables
\begin{enumerate}
\item {\tt  VERIFICARLO\_MCAMODE}:
\begin{enumerate}
\item MCA: (default value) {\it Monte Carlo Arithmetic} with inbound and outbound error
\item IEEE: noiseless execution equivalent to IEEE (useful to check that the instrumentation didn't 'break' the code).
\item PB: {\it Precision Bounding} only inbound error
\item RR: {\it Random Rounding} only outbound error
\end{enumerate}
\item {\tt VERIFICARLO\_PRECISION} control the virtual precision, {\it i.e.} the magnitude of the noise introduction. The default value is  53, the less significant bit of double precision floating point number. The equivalent in single precision is 24.~\cite{denis2016verificarlo,parker1997monte}.

\item {\tt VERIFICARLO\_BACKEND} In its github released version, Verificarlo supports two main {\it backends} with theoretically equivalent behavior:
\begin{description}
\item[{\tt MPFR}] : This is the original reference implementation using the multi precision library MPFR~\cite{Fousse:2007:MMB:1236463.1236468} and extrapolated from  Frechtling~\cite{frechtling2015automated} work.

\item[{\tt QUAD}] : Since doubling the mantissa number of bits is enough to represent all the noise that can influence the rounding, {\tt Quad} uses {\it ad-hoc} quadruple precision (binary128) implementation of the stochastic arithmetic for double precision (binary64) computation, and double (binary64) for single precision (binary32) computation.
\end{description}
\end{enumerate}

To change these variable values, you can use the  {\tt export} Unix command, or do an affectation in your shell script running the application, or using the API changing it dynamically from the application source code.

In order to be sure of the execution setup during your experiment, we advise you to fix the desired parameters in the execution scripts.


\subsubsection{From the source code}

Using the verificarlo API, it is possible for advanced user to control verificarlo at runtime from the application by inserting calls into the program.

This functionality is undocumented and we encourage users who want to try to do it to contact us directly from github.



\section{Practical exercice: Polynomial evaluation}

Polynomial evaluation is a common source of computational error. Polynomials are frequently used for function interpolation in libraries or user codes. As we will see, different evaluations of the same polynomial do not have the same behavior in terms of performance or numerical accuracy.

This tutorial is uses the following Tchebychev polynomial from ~\cite[pp.52-54]{parker1997monte}:

$$T(x)=\sum_{i=0}^{10}{a_i \times x^{2i}}$$
With:
$a_i \in [
  1,
  \matminus 200,
  6600,
  \matminus 84480,
  549120,
  \matminus 2050048,
  4659200,
  \matminus 6553600,
  5570560,
  \matminus 2621440,
  524288
]$

We are interested in evaluating  $T$ near $1$.
This example is discussed with details in~\cite[pp.52-54]{parker1997monte}.

\subsection{Expanded form evaluation}

\subsubsection{First steps with Verificarlo}

In this first approach, we will evaluate the polynomial in its expanded form as given in the previous section. We first evaluate it in single precision.

\begin{question}
  \begin{enumerate}[(a)]
  \item Open the {\tt tchebychev.c} file and observe the function {\tt REAL expanded(REAL x)}.

  \item Compile {\tt tchebychev.c} with {\tt verificarlo} using the following command:
\begin{verbatim}
verificarlo -D FLOAT tchebychev.c -o tchebychev
\end{verbatim}
  \item Run the program.
  \end{enumerate}
\end{question}

When running the program it complains that at least one backend should be
selected with the \texttt{VFC\_BACKENDS} environment variable.
To run the program with standard IEEE-754 arithmetic one can use the simple
IEEE backend (\texttt{libinterflop\_ieee.so}). The backend has a \texttt{--debug}
option that shows the floating-point operations that have been instrumented.

\begin{question}
Run the program using the IEEE backend,
\begin{verbatim}
VFC_BACKENDS="libinterflop_ieee.so --debug" ./tchebychev 0.99 EXPANDED
\end{verbatim}
\end{question}

To evaluate the numerical error we will use the Monte Carlo arithmetic backend
(\texttt{libinterflop\_mca.so}.  Because we are using IEEE-754 floats, to
simulate round-off errors we should work at a 24 bits precision.

\begin{question}
  \begin{enumerate}[(a)]
  \item Run the program using the Monte Carlo Arithmetic backend at 24 bits precision,
\begin{verbatim}
VFC_BACKENDS="libinterflop_mca.so --precision=24" ./tchebychev 0.99 EXPANDED
\end{verbatim}
  \item Execute the program multiple times. What can you observe?
  \end{enumerate}
\end{question}

Monte Carlo arithmetic backend supports different modes,
\begin{itemize}

  \item \texttt{-{}-mode=rr} is the \emph{random round} mode that adds noise on the
    result of an operation only when the operation is not exactly representable
    at the given precision. This mode is useful to simulate the effect of
    round-off errors.

  \item \texttt{-{}-mode=pb} is the \emph{precision bound} mode that adds noise on
  the operands before performing the operation. It is useful to simulate the
    effect of cancellations errors.

  \item \texttt{-{}-mode=mca} is the default mode that combines \texttt{rr} and
  \texttt{pb} modes.

\end{itemize}

\begin{question}
  \begin{enumerate}[(a)]
  \item Now recompile with verificarlo the program in double precision using the command
    {\tt verificarlo -D DOUBLE tchebychev.c -o tchebychev} \\
  \item Execute the program with arguments \texttt{0.99 EXPANDED} with the Monte Carlo arithmetic backend. Try different precisions such as 53, 24, 10. Try also to use different modes (rr, pb, mca).
  \end{enumerate}
\end{question}

\subsubsection{Numerical quality analysis}

In this section, we propose you to analyze the numerical quality of the results computed by the expanded evaluation of the polynomial. To simplify this task, you can use the script~\texttt{run.sh} which automates the required verificarlo runs. Visualization is done using the \texttt{plot.py} script.


\begin{question}
  \begin{enumerate}[(a)]
 \item Open {\tt run.sh} and analyze how it works. We will emulate the single precision using the virtual precision of verificarlo.
  \item Modify {\tt run.sh} to evaluate the polynomial in the interval $[0.5,1]$ by $0.001$ step.
  \item Open {\tt plot.py} and analyze how it works, in particular the data that will be plotted.
  \end{enumerate}
\end{question}









The \texttt{plot.py} script generates plot similar to
figure~\ref{fig:expanded:double:24}.
The upper part of the figure represents the number $s$ of significant digits of the results: $s=-\log_{10}\left|\dfrac{\hat\sigma}{\hat\mu}\right|$ with $\hat\sigma$ the sample empirical standard deviation  $\hat\mu$ their average.


The central part is the empirical standard deviation $\hat\sigma$ for each value of $x$.

Finally the lowest parts are the $T(x)$ samples and their average in dotted line. The 20 Monte Carlo samples $T(x)$ are plotted for each $x$ value (sometime overlapping on the graphic)

\begin{question}
\begin{enumerate}[(a)]
\item To execute the {\tt EXPANDED} version with {\tt DOUBLE} and a virtual precision of 24 bits, execute the command: {\tt ./run.sh EXPANDED
      DOUBLE 24 }. \newline This command's output is given in figure~\ref{fig:expanded:double:24}.
  \item With a virtual precision of 53, execute the command: {\tt ./run.sh EXPANDED DOUBLE 53} \newline
  This command's output is given in figure~\ref{fig:expanded:double:53}..
  \end{enumerate}
\end{question}

\begin{figure}[h]
\center \includegraphics[width=.8\textwidth]{EXPANDED-DOUBLE-24.pdf}
  \caption{Evaluation of T(x) in its expanded form, compiled in double precision, with a virtual precision of 24}
  \label{fig:expanded:double:24}
\end{figure}
\begin{figure}[h]
\center \includegraphics[width=.8\textwidth]{EXPANDED-DOUBLE-53.pdf}
  \caption{Evaluation of T(x) in its expanded form, compiled in double precision, with a virtual precision of 53}
  \label{fig:expanded:double:53}
\end{figure}

Close to 1, the polynomial evaluation is subject to {\it cancellations} which rapidly decrease the result precision. The double precision on the contrary seems satisfactory.

However using double precision is just moving the problem closer to 1 and it forces the programmer to use a larger and more costly data type.

Nevertheless if the user is already using double precision number, and if the precision is still not satisfactory, how to solve the issue? Or what if we need to use single precision?

\FloatBarrier

\subsection{Evaluation using Horner scheme}

It exists many other ways to evaluate polynomials, using associativity, commutativity and factorization. More often they are explored for sake of performance, but they also greatly influence the precision of the evaluation. One of them reputed to both performance with good numerical behavior is the Horner scheme which for our polynomial correspond to the following form:
% Horner reputed for good numerical behavior?

\[
	T(x) = (\dots((a_n\times x^2 + a_{n-1})\times x^2 + a_{n-2})\dots) \times x^2
    + a_0
\]

$$T(x) = (((((((((524288*x^2-2621440)*x^2+5570560)*x^2-6553600)*$$
$$x^2+4659200)*x^2-2050048)*x^2+549120)*x^2-84480)*$$
$$x^2+6600)*x^2-200)*x^2+1$$

\begin{question}
  \begin{enumerate}[(a)]
  \item Open the file {\tt tchebychev.c} and have a look to the function {\tt REAL horner(REAL x)}
\item While keeping previous execution parameters, execute the command {\tt ./run.sh HORNER DOUBLE 53}.  \newline The output of this command is given in figure~\ref{fig:horner:double:53}.
  \end{enumerate}
\end{question}

\begin{question}
 \item Modify the {\tt run.sh} script to evaluate the polynomial from $0.5$ to $1$ by $0.001$.
\item Execute the command {\tt ./run.sh HORNER DOUBLE 24}  \newline
The output of this command is given in figure~\ref{fig:horner:double:24}.

\end{question}

As shown in this experiment, the Horner scheme has a limited influence on the result precision ($\simeq$ 1 more bit). However, it minimizes the number of operations and allows to use the FMA ({\it Fused Multiply Add}). For a polynomial of degree $n$, it produces $n-1$ FMA. Moreover, when doing multiple independent evaluations it can be vectorized.


\begin{figure}[htb]
\center \includegraphics[width=.8\textwidth]{HORNER-DOUBLE-53.pdf}
  \caption{Evaluation of T(x) using Horner scheme, compiled in double precision, with a virtual precision of 53}
  \label{fig:horner:double:53}
\end{figure}
\begin{figure}[htb]
\center \includegraphics[width=.8\textwidth]{HORNER-DOUBLE-24.pdf}
  \caption{Evaluation of T(x) using Horner scheme, compiled in double precision, with a virtual precision of 24}
  \label{fig:horner:double:24}
\end{figure}


\FloatBarrier

\subsection{Factored form}

We will now evaluate the evaluation precision of the following factored rewriting:
\[
	T(x) = 1 + 8x^2\,(x-1)\,(x+1)\,(4x^2 + 2x - 1)^2\, (4x^2 - 2x - 1)^2\,(16x^4 - 20x^2 + 5)^2
\]

\begin{eqnarray*}
T(x) &=& 8.0*x^2*(x - 1.0)*(x + 1.0) \\
 & & * (4.0*x^2 + 2.0*x - 1.0)*(4.0*x^2 + 2.0*x - 1.0) \\
 & & * (4.0*x^2 - 2.0*x - 1.0)*(4.0*x^2 - 2.0*x - 1.0)* \\
 & & * (16.0*x^4 - 20.0*x^2 + 5.0)*(16.0*x^4 - 20.0*x^2 + 5.0) + 1 \\
\end{eqnarray*}

\begin{question}
  \begin{enumerate}[(a)]
  \item Open the file {\tt tchebychev.c} and have a look to the function {\tt REAL factored (REAL x)}
\item Execute the command {\tt ./run.sh FACTORED DOUBLE 24} \newline
The output of this command is given in figure~\ref{fig:factored:double:24}.
\item Compare these results to those obtained with EXPANDED and HORNER versions.
  \end{enumerate}
\end{question}

\begin{figure}[h]
\center \includegraphics[width=.8\textwidth]{FACTORED-DOUBLE-24.pdf}
  \caption{Evaluation of T(x) in its factored form, compiled in double precision, with a virtual precision of 24}
  \label{fig:factored:double:24}
\end{figure}

\begin{question}
  Explain what happens when $T(x)=1$ for $x\simeq 0.6$,
  $x\simeq 0.8$ et $x\simeq 0.95$.\\~\\
  $\rightarrow$ It is an example where the error is absorbed and the precision and accuracy  of the results are improved.
\end{question}


\begin{question}
  \begin{enumerate}[(a)]
    \item Modify the {\tt run.sh} script to evaluate the  polynomial between $0.99$ and $1$ by $0.00001$ step.
  \item Run the scripts to execute and visualize the results for
    FACTORED, EXPANDED and HORNER with a virtual precision of 53. The results are respectively presented in figure~\ref{fig:factored:double:53:zoom},\ref{fig:expanded:double:53:zoom}
    and~\ref{fig:horner:double:53:zoom}.

\item Reproduce the result with a virtual precision of 24. The results are respectively presented in figure~\ref{fig:factored:double:24:zoom},\ref{fig:expanded:double:24:zoom}
    and~\ref{fig:horner:double:24:zoom}

\end{enumerate}
\end{question}

\begin{figure}[h]
  \center \includegraphics[width=.8\textwidth]{FACTORED-DOUBLE-53-zoom.pdf}
  \caption{Evaluation of T(x) in its factored form, compiled in double
    precision, with a virtual precision of 53}
  \label{fig:factored:double:53:zoom}
\end{figure}

\begin{figure}[h]
  \center \includegraphics[width=.8\textwidth]{EXPANDED-DOUBLE-53-zoom.pdf}
  \caption{Evaluation of T(x) in its expanded form, compiled in double
    precision, with a virtual precision of 53}
  \label{fig:expanded:double:53:zoom}
\end{figure}

\begin{figure}[h]
  \center \includegraphics[width=.8\textwidth]{HORNER-DOUBLE-53-zoom.pdf}
  \caption{Evaluation of T(x) using Horner scheme, compiled in double precision,
    with a virtual precision of 53}
  \label{fig:horner:double:53:zoom}
\end{figure}

\begin{figure}[h]
  \center \includegraphics[width=.8\textwidth]{FACTORED-DOUBLE-24-zoom.pdf}
  \caption{Evaluation of T(x) in its factored form, compiled in double
    precision, with a virtual precision of 24}
  \label{fig:factored:double:24:zoom}
\end{figure}

\begin{figure}[h]
  \center \includegraphics[width=.8\textwidth]{EXPANDED-DOUBLE-24-zoom.pdf}
  \caption{Evaluation of T(x) in its expanded form, compiled in double
    precision, with a virtual precision of 24}
  \label{fig:expanded:double:24:zoom}
\end{figure}

\begin{figure}[h]
  \center \includegraphics[width=.8\textwidth]{HORNER-DOUBLE-24-zoom.pdf}
  \caption{Evaluation of T(x) using Horner scheme, compiled in double precision,
    with a virtual precision of 24}
    \label{fig:horner:double:24:zoom}
\end{figure}

\subsection{Conclusion}


From a general stand point, for every arithmetic expression in a program, it exists many valid rewriting. They are not all equivalent in terms of performance, precision, and accuracy!

\begin{question}
What is the best approach?
\end{question}

\FloatBarrier



\input{compensated}

\section{Using Veritracer}

During the first public presentation, J.M. Muller asked us if our tool could handle the following case illustrated in his book:

 \begin{equation}
   u_{n} = 111 - \dfrac{1130}{u_{n-1}} + \dfrac{3000}{u_{n-1}u_{n-2}}
   \label{eq:muller_sequence_un}
 \end{equation}

The fixed points of this sequence are the roots of the polynomial:

\quad $u^3 - 111u^2 + 1130u - 3000 = (u-5)(u-6)(u-100)$

With the chosen initial values, $u_0=2$ and $u_1=-4$, the mathematical answer is 6.

\subsection{Running the code with Veritracer}

We will first reproduce the experiment of D. Stott Parker~\cite{parker1997monte}.

\begin{question}
    \begin{enumerate}[(a)]
\item Compile and test {\tt muller.c}
\item Modify {\tt Makefile} to compile with {\tt verificarlo}
\item Collect 32 results with virtual precision 53.
\item Compute the number of significant digits $s$.

$\Rightarrow$ As you can see the program is converging to the dominant root of the polynomial {\it i.e.} 100, with maximum precision! Therefore, by looking only to the final precision one could conclude that this is the correct answer\\~\\

\end{enumerate}
\end{question}

We will now do the experiment with veritracer to better understand this result.

\begin{question}
    \begin{enumerate}
        \item Type the command {\tt verificarlo -{}-help} to print how to call veritracer usage
        \item Type the command, and check the output and the {\tt .map} generated files \newline
        {\tt verificarlo -{}-verbose -{}-tracer muller.c -o muller -{}-function muller1 }
        \item Remove the current location map and launch veritracer in backtrace mode with the following command: {\tt verificarlo -{}-verbose -{}-tracer muller.c -o muller -{}-function muller1  -{}-tracer-backtrace}
        \item Run the program in the tracer environment with the following command: \newline
        {\tt veritracer launch -{}-force -{}-binary muller -{}-jobs 29}
        \item Check the content of the {\tt .vtrace} directory
        \item To launch the trace analysis run the command: {\tt veritracer analyze } and check the result in  the file {\tt .vtrace/veritracer.000bt}
        \item Plot the result with the provided {\it ad-hoc} script with the following command: \newline
        {\tt veritracer plot .vtrace/veritracer.000bt}
        \item It is possible to add invocation information with the following command: \newline
        {\tt veritracer plot .vtrace/veritracer.000bt -{}-invocation-mode}
        \item and basic statistics: {\tt veritracer plot .vtrace/veritracer.000bt -{}-invocation-mode -{}-mean -{}-std}
%rm locationInfo.map
%verificarlo -{}-verbose -{}-tracer muller.c -o muller -{}-function muller1  -O3
%locationInfo.map -> vide Why ?

%verificarlo -{}-verbose -{}-tracer muller.c -o muller -{}-function muller1  -O3
%verificarlo -{}-verbose -{}-tracer muller.c -o muller -{}-function muller1
%-{}-tracer-backtrace -O3 -{}-tracer-level temporary
%cat locationInfo.map


%veritracer launch -{}-force -{}-binary muller -j 29
%veritracer analyze
%empty traces ? Pourquoi


%verificarlo -{}-verbose -{}-tracer muller.c -o muller  -{}-tracer-backtrace
%-O3 -{}-tracer-level temporary
    \end{enumerate}
    $\Rightarrow$ NEW update: a pre-release GUI to plot and navigate in the trace is available on github for a more friendly user experience
\end{question}

$\Rightarrow$ Figure~\ref{fig:muller_sequence_un} has been automatically generated thanks to veritracer. It shows the evolution of the significant digits with the iteration. We can observe a gradual degradation of the precision until it reaches no significant digits. Then the precision is gradually improving to reach the maximum attainable in double precision format, {\it i.e.} 17. This plot allow us to conclude that the generated results, while being precise, as lost all its accuracy.

For the final experiment of this section we will use veritracer on the Tchebychev polynomial from this tutorial.

\begin{question}
  \begin{enumerate}[(a)]
      \item Experiment veritracer on the Tchebychev Polynomial evaluation between $0$ and $1$ by $0.001$.
      $\Rightarrow$ The result is presented in figure~\ref{fig:veritcheby}.

%rm locationInfo.map
%verificarlo -{}-tracer tchebychev.c -o tchebychev -O3  -{}-tracer-level
%temporary -{}-tracer-backtrace
%cat locationInfo.map

%veritracer launch -{}-force -{}-binary "./tchebychev FACTORED 100" -j 29 &&
%veritracer analyze

%veritracer plot .vtrace/veritracer.000bt -{}-invocation-mode -{}-mean -{}-std
%
%cat locationInfo.map |grep factored| grep ret
%veritracer plot .vtrace/veritracer.000bt -{}-invocation-mode -{}-mean -{}-std
%-v $$

%cat locationInfo.map |grep horner| grep ret
%veritracer launch -{}-force -{}-binary "./tchebychev  HORNER 100" -j 29 &&
%veritracer analyze
%veritracer plot .vtrace/veritracer.000bt -{}-invocation-mode -{}-mean -{}-std
%-v $$

  \end{enumerate}
\end{question}

 \begin{figure}[h!]
  \centering\includegraphics[width=0.8\linewidth]{muller_sequence_un.jpg}
  \caption{The evolution of the number of significant decimal digits ($s$) over time
     for the sequence $u_n$ in equation~\ref{eq:muller_sequence_un}.
    For $n=14$, $s$ is below 0 means that $u_{14}$ has no correct decimal digits.
     Only checking the final results is not enough to detect accuracy loss.
  }

 \label{fig:muller_sequence_un}
 \end{figure}
%
%
\begin{figure}[h!]
  \centering\includegraphics[width=1\linewidth]{tcheby_veritracer_large_font.jpg}
 \caption{The evolution of the number of significant decimal digits ($s$) over time
     for the evaluation of the tchebytchev polynomial used in this tutorial, from $0$ to $1$ by $0.001$.
  }
 \label{fig:veritcheby}
 \end{figure}
%
%
 ~\\~\\$\Rightarrow$ Veritracer contextualizes and traces variable precision over time: it helps to understand the arithmetic behavior of a program. It allows to focus program analysis on a limited set of functions, variables and inputs context.
%
%\FloatBarrier
%\clearpage



\newpage
\bibliographystyle{ieeetr}
\section{Bibliography}
\bibliography{bibliography}

\end{document}
