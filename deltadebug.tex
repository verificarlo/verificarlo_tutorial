\section{Pinpointing bugs with Delta-Debug: Archimedes method}

\begin{figure}
  \centering
  \begin{tikzpicture}[scale=1.5]
    \draw [magenta] (0,0) circle (1);
    \draw [fill, black] (0,0) circle (.02);
    \begin{scope}[yshift=1cm]
    \draw [blue, turtle={ home,
      right=90, forward=.5773502691896257, % tan(pi/6)
      right=60, forward=2*.5773502691896257,
      right=60, forward=2*.5773502691896257,
      right=60, forward=2*.5773502691896257,
      right=60, forward=2*.5773502691896257,
      right=60, forward=2*.5773502691896257,
      right=60, forward=.5773502691896257
    }];
    \end{scope}
\end{tikzpicture}
  \caption{Archimedes method to approximate PI with a 6-sided circumscribed polygon.
    \label{fig:archimedes}
  }
\end{figure}

In this section we demonstrate how we can use Verificarlo to precisely localize a numerical bug in a program. The localization method is based on the Zeller's Delta-Debug reduction method~\cite{zeller2001automated}. Verificarlo uses the Interflop's stochastic Delta-Debug library by Bruno Lathuilière.

In 200BC Archimedes proposed the first numerical method for computing $\pi$.
Archimedes method uses one $6.n$-sided circumscribed polygon to the unit circle
whose area provides an upper bound for $\pi$ and one $6.n$-sided inscribed polygon
whose area provides a lower bound for $\pi$.

Here we will use the circumscribed polygon to approximate $\pi$.
Figure~\ref{fig:archimedes} shows a 6-sided circumscribed polygon to the unit
circle. Archimedes shows geometrically that the area of the polygons can be computed
with the following recursive sequence,

\begin{align*}
  T_1 &= \frac{1}{\sqrt{3}} \\
  T_{i+1} &= \frac{\sqrt{T_i^2+1} - 1}{T_i} \\
  A_{i} &= 6 \times 2^{i-1} \times T_{i} \xrightarrow[i \to \infty]{} \pi
\end{align*}









